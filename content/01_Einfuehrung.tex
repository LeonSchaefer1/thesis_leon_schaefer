\chapter{Einführung}
\label{sec:einfuehrung}

Die Spieleindustrie ist im Wandel. Reine Singleplayer-Spiele, die ohne andere menschliche Mitspieler auskommen, werden immer seltener entwickelt. Die Erfolgsaussichten erscheinen um ein Vielfaches schlechter. Die drei bestplatzierten Titel der meistverkauften PC- und Konsolenspiele 2019 (FIFA 2020, Call of Duty: Modern Warfare und Mario Kart 8 Deluxe) in Deutschland enthalten einen großen Multiplayer-Anteil \cite{gameVerbandderdeutschenGamesBranchee.V..2020}. Deutlicher wird der Trend, wenn die Statistik der meistgespielten Titel auf der Internet Vertiebsplattform für Computerspiele Steam betrachtet wird. Das mit Abstand meistgespielte Spiel 2020 ist das Multiplayer-First-Person-Shooter Spiel 'Counter-Strike: Global Offensive' gefolgt von 'DOTA 2' auf Platz 2. Beide Spiele sind primär Multiplayer-Titel \cite{GitHyp.February2021}.

Einer internationalen Umfrage zufolge geben 60 \% der Befragten an, seit Beginn der Corona-Pandemie mehr Multiplayer-Videospiele zu spielen \cite{SimonKucher&Partners.2020}. Wegen der Marktentwicklung ist es für angehende Entwickler, welche in der Spieleindustrie Fuß fassen möchten, auf lange Sicht eine gute Entscheidung, sich mit der Entwicklung von Multiplayer-Spielen auseinanderzusetzen.

\section{Problembeschreibung und Motivation}

Konzeption und Entwicklung von Online-Multiplayer-Spielen stellt angehende Entwickler vor große Herausforderungen. Einer der Hauptunterschiede zwischen einem Multiplayer- und einem Singleplayer-Spiel ist, dass eine Spielumgebung unabhängig von einem Spieler existiert. Das bedeutet, dass sich in der Regel bei Singleplayer-Spielen alle Software-Prozesse bei Beendigung des Spiels auch schließen, bei Multiplayer-Spielen jedoch laufen Client- und Serverprozess unabhängig voneinander. So ist es möglich, dass Serverprozesse, und somit auch Spiellogik unabhängig von anwesenden Clients (Spieler) weiterläuft und sich Zustände einer Spiel-Session verändern. 

Außerdem benötigt das Versenden von Nachrichten über das Internet eine gewisse Zeit. Der Entwickler muss also herausfinden, wie man mit asynchroner Kommunikation, Clients und Servern umgeht. Ebenso stoßen Entwickler vor Probleme wie Hosting, Matchmaking und Datenverwaltung.
\cite{Payne.18.09.2019}

Einheitliche Vorgehensmodelle und Best Practises gibt es nicht oder beziehen sich stets auf eine spezielle Art von Spieltyp. Das Thema Hosting ist ebenfalls ein komplexer Aspekt eines solchen Projekts, welcher von Anfängern oft nicht sofort durchschaut wird. Durch erste Projekterfahrung gewinnen Entwickler nach und nach das Knowhow. Ohne ein erfahrenes Team sind die Chancen auf Misserfolg jedoch hoch. \cite{Payne.18.09.2019}

Die Motivation für diese Arbeit war es, angehenden Entwicklern eine Einstiegshilfe in die Entwicklung von Multiplayer-Spielen zu geben. Hierzu wurden Konzepte erarbeitet, die sie in ihrem ersten Praxisprojekt ausprobieren und idealerweise davon profitieren können.

\section{Ziele dieser Arbeit}

Das primäre Ziel dieser Arbeit ist es, Anfängern den Einstieg in die Entwicklung von Multiplayer-Spielen zu erleichtern. Ein Entwickler soll außerdem nach dem Lesen dieser Arbeit eine konkrete Vorstellung davon besitzen, welche Voraussetzungen ein Framework oder eine Game-Engine erfüllen muss, damit die Konzepte möglichst leicht umzusetzen sind.

Die Arbeit soll als eine 'Blaupause' dienen, welche andere Entwickler nutzen können, um einen leichteren Einstieg in ein Multiplayer-Projekt zu gewährleisten. Diese Blaupause soll generisch sein, sodass sie unabhängig von einem konkreten Implementierungskontext funktioniert. Die beschriebenen Konzepte sollen auf möglichst viele Szenarien anwendbar sein. Durch die Konzepte sollen verschiedene Use cases in der Multiplayer-Spieleprogrammierung abgedeckt werden. Beispiele sind das Aufbauen und Trennen von Client/Server Verbindungen, Matchmaking \cite{Wikipedia.2021b} und Spielerverwaltung. Als 'Proof of Concept' soll ein Prototyp eines Hide-and-Seek Multiplayer-Spiels dienen, welches sich an den abstrahierten Konzepten orientiert, bzw. diese implementiert. 

Folgende Forschungsfragen soll die Arbeit beantworten:

\begin{itemize}
	\item \label{RQ1} RQ1: Mit welchen konkreten Problemen werden angehende Entwickler von Multiplayer-Spielen konfrontiert?
	\item \label{RQ2} RQ2: Wie können diese Probleme im Einzelnen durch gelöst werden?
	\item \label{RQ3} RQ3: Wie könnte ein Studienkonzept aussehen, welches bei Durchführung beweist, dass die in dieser Arbeit beschriebenen Konzepte einen Mehrwert bei der Entwicklung eines Multiplayer-Spiels für angehende Entwickler liefern?
\end{itemize}

\section{Related Work}

Ähnliche Ansätze, welche versuchen die komplexen Anforderungen an ein Multiplayer-Spiel zu reduzieren finden sich beispielsweise in einem Zeitungsartikel "Making Multiplayer Games Doesn’t Have to Be Difficult" von von John Payne aus 2019  \cite{Payne.18.09.2019}. In diesem beschreibt Payne, wo die Schwierigkeiten in der Entwicklung von Multiplayer-Spielen liegen und wie er, und sein Entwicklerteam mithilfe eines eigenen Frameworks diese Probleme reduzieren.

Der Artikel von Payne war die Motivation ein modernes Framework für den Prototypen zu wählen welcher im Rahmen dieser Arbeit entwickelt wurde. Außerdem wurde der Artikel als Orientierung genutzt, um bei der Wahl des Frameworks eine fundierte Entscheidung treffen zu können.

Im Forschungsartikel "High-Level Development of Multiserver Online Games" von Frank Glinka aus dem Jahr 2008 \cite{Glinka.2008} wird ebenfalls ein Framework beschrieben, welches Entwickler unabhängig von einer konkreten Entwicklungsumgebung benutzen können. Dieses Framework beinhaltet die Implementierung mehrerer 'Manager'. Diese erledigen bestimmte Aufgaben auf Seiten des Servers, beispielsweise die Verwaltung der Spiel-Zonen, in welchen sich Spieler aufhalten, die Verwaltung der Spieler selbst oder die Nachrichten, welche sich Spieler gegenseitig schicken können. Dieses Framework richtet sich hauptsächlich nach sog. 'massively multiplayer online games' (MMOG), also Spielen, in denen eine große Anzahl an Spielern (1000+) auf einer Serverinstanz spielen.

Der Artikel von Glinka war auch die Inspiration dafür, die Konzepte, welche in dieser Arbeit aufgestellt wurden, zu erarbeiten. Da sich die 'Manager' Konzepte, welche in Glinka's Framework aufgeführt sind, lediglich auf Server-interne Aufgaben beschränken wurden die Konzepte dieser Arbeit ebenfalls auf clientseitige Aufgabenfelder ausgeweitet.

\section{Vorgehensweise}

\hyperref[RQ1]{RQ1} soll wie folgt beantwortet werden: \\
Anhand von Literatur- und Internetrecherche werden die konkreten Probleme, vor welchen angehende Entwickler stehen, zusammengetragen. Diese dienen als Grundlage für die Beantwortung von \hyperref[RQ2]{RQ2}.

\hyperref[RQ2]{RQ2} soll wie folgt beantwortet werden: \\
Aus der bisher gefundenen Literatur und gesammelten Praxiserfahrung werden Konzepte abstrahiert, welche jeweils einen bestimmten Einsatzzweck erfüllen sollen. Hierbei wurde auf Implementierungsdetails verzichtet und lediglich ein generelles Konzept erarbeitet, an welchem der Entwickler sich während des Implementierungsprozesses orientieren kann. Jedes dieser Konzepte wurde in einem Prototyp umgesetzt, welche im Kapitel \hyperref[sec:realisierung]{'Realisierung'} näher vorgestellt wird.

\hyperref[RQ3]{RQ3} soll wie folgt beantwortet werden: \\
Es wird ein Studienkonzept erarbeitet, welches beschreibt, wie durch wissenschaftliche Methoden und Forschung untersucht werden kann, ob die Konzepte einen tatsächlichen Mehrwert für angehende Entwickler bietet. Das Studienkonzept findet sich im Kapitel \hyperref[sec:konzepte]{'Konzepte'} in der Sektion \hyperref[studienkonzept]{'Studienkonzept'}.

Im Anschluss erfolgt eine Evaluation der Ergebnisse, die Stärken und Schwächen des methodischen Vorgehens werden realistisch eingeschätzt.

\section{Abgrenzung}

Die Arbeit ist keine Schritt-für-Schritt-Anleitung für Multiplayer-Spielprojekte. Implementierungsdetails müssen von einem Entwickler, welcher diese Arbeit als Hilfsmittel nutzt, selbstständig konzipiert und umgesetzt werden. Lediglich wird auf Spielkonzepte Rücksicht genommen, welche in der heutigen Industrie gängig sind \cite{Wikipedia.2021b}. Es ist durchaus denkbar, dass zukünftige Multiplayer Architekturen nicht mehr mit den hier beschriebenen Konzepten umsetzbar sind.

Es ist zudem nicht final bewiesen, dass die beschriebenen Methoden tatsächlich einen Mehrwert bei unterschiedlichen Projekten bieten. Die Konzepte wurden lediglich an einem Prototyp getestet und validiert. Um zu testen, ob die Konzepte auch praxistauglich für verschiedene Arten von Projekten sind, muss eine weitergehende Untersuchung erfolgen. Diese kann anhand des in dieser Arbeit beschriebenen \hyperref[studienkonzept]{Studienkonzept} stattfinden. Die erarbeiteten Konzepte sind unabhängig von einer Game-Engine oder einem Framework. Es aber möglich, dass ein bestehendes Framework oder eine Game-Engine Hilfestellung bei der Implementierung der Konzepte bietet, oder diese bereits als Feature bereitstellt.

\section{Struktur der Arbeit}

Diese Arbeit besteht aus der \hyperref[sec:einfuehrung]{Einführung}, in welcher die generelle Problemstellung erläutert, und die Forschungsfragen gestellt wurden. Ebenfalls wurde die Relevanz des Themengebiets verdeutlicht und eingeordnet. Im Anschluss wird der \hyperref[sec:hintergrund]{Hintergrund} erklärt, dort sind alle Informationen zu finden, um die späteren Konzepte zu verstehen. Hier werden ebenfalls Begrifflichkeiten und Grundlagen erläutert. Das Kapitel \hyperref[sec:konzepte]{'Konzepte'} behandelt nun die erarbeiteten Konzepte, welche in Summe die oben beschriebene Blaupause abbilden. Es wird auf jedes einzelne Konzept im Detail eingegangen und genauer erklärt. Anschließend werden die beschriebenen Konzepte in einem Prototyp umgesetzt, dieser wird im Kapitel \hyperref[sec:realisierung]'{Realisierung'} näher erklärt. In diesem Kapitel befindet sich kommentierter Beispielcode aus dem umgesetzten Prototyp, um einen besseren Eindruck der gelösten Probleme zu bekommen. Es folgt das Kapitel \hyperref[studienkonzept]{'Studienkonzept'}, in welchem Möglichkeiten zur weiterführenden Forschung aufgestellt werden. Im Kapitel \hyperref[sec:abschluss]{'Abschluss'} wird eine Zusammenfassung und Diskussion der Arbeit, sowie ein Fazit, die nächsten Schritte und ein Ausblick auf mögliche zukünftige Entwicklungen aufgezeigt.
