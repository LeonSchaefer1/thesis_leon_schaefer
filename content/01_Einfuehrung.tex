\chapter{Einführung}

\section{Problembeschreibung}

Konzeption und Entwicklung von Online-Multiplayerspielen stellt angehende Entwickler vor große Herausforderungen. Die Einstiegshürden sind groß, der Dschungel an Technologien und Frameworks unübersichtlich. Einheitliche Vorgehensmodelle und Best-Practises gibt es nicht oder beziehen sich stets auf eine spezielle Art von Spieltyp. Das Thema Hosting ist ebenfalls ein komplexer Aspekt eines solchen Projekts, welcher von Anfängern oft nicht sofort durchschaut wird.

\section{Motivation}

Angetrieben wurde diese Arbeit durch den eigenen Wunsch professionell ein Multiplayer-Spielprojekt umzusetzen und den Grundstein für ein einheitliches Vorgehen für zukünftige Projekte zu legen. Von diesem Grundstein sollen möglichst viele weitere Projekte profitieren.

\section{Ziele dieser Arbeit}

Die Arbeit soll eine Blaupause liefern, welche andere Entwickler nutzen können, um einen leichteren Einstieg in ein Multiplayer-Projekt zu gewährleisten. Diese Blaupause soll allerdings derart generisch sein, dass sie unabhängig von einem konkreten Implementierungs-Kontext funktioniert. Die beschriebenen Konzepte sollen auf möglichst viele Szenarien anwendbar sein.

Als Proof-of-Concept soll die eigene Implementierung eines Hide-and-Seek Multiplayer-Spiels dienen, welches sich an den abstrahierten Konzepten orientert, bzw. diese implementiert.

\section{Abgrenzung}

Die Arbeit ist jedoch keine Schritt-für-Schritt Anleitung. Implementierungsdetails müssen vom Benutzer selbstständig konzipiert und umgesetzt werden. Es wird außerdem lediglich Rücksicht genommen auf Spielkonzepte, welche in der heutigen Industrie gängig sind. Es ist durchaus denkbar, dass zukünftige Multiplayer / Matchmaking Architekturen nicht mehr mit den hier beschriebenen Konzepten umzusetzen sind.


\section{Vorgehensweise}

Aus der bisher gesammelten Praxiserfahrung werden Konzepte abstrahiert, welche einen bestimmten Einsatzzweck erfüllen sollen. Hierbei wurde auf Implementierungsdetails verzichtet und lediglich ein generelles Konzept erarbeitet, an welchem der Entwickler sich während des Implementierungsprozesses orientieren kann.

\section{Struktur der Arbeit}

Dies Arbeit besteht aus der Einführung, wo die generelle Problemstellung erläutert, und die Forschungsfragen gestellt wurden. Im Anschluss wird der Hintergrund erklärt, dort sind alle Informationen zu finden, um die späteren Konzepte zu verstehen. Diese werden als nächstes Kapitel behandelt. Anschließend werden die beschriebenen Konzepte in einer Beispielimplementierung umgesetzt, diese bildet das Kapitel "Realisierung". In der Zusammenfassung werden die Ergebnisse der Arbeit beleuchtet und ein Ausblick aufgezeigt.
