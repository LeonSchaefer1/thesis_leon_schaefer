% Wie schreibe ich das Abstract?
% Bei der Vorbereitung des Abstracts sollten Sie sich an folgenden grundlegenden Punkten orientieren:

%% Motivation des Textes:
%% worin liegt die Bedeutung der entsprechenden Forschung, warum sollte der längere Text gelesen werden?

%% Fragestellung:
%% welche Fragestellung(en) versucht der Text zu beantworten, was ist der Umfang der Forschung, was sind die zentralen Argumente und Behauptungen?

%% Methodologie:
%% welche Methoden/Zugänge nutzt der Autor/die Autorin, auf welche empirische Basis stützt sich der Text?

%% Ergebnisse:
%% zu welchen Ergebnissen kam die Forschung, was sind die zentralen Schlussfolgerungen des Textes?

%% Implikationen:
%% welche Schlussfolgerungen ergeben sich aus dem Text für die Forschung, was fügt der Text unserem Wissen über das Thema hinzu?
 
\begin{abstract}
Ideensammlung:

1. Grundlagen des Mirror Frameworks:
--> worauf basiert es eigendlich? 
--> Features erklären.
--> Implementierungsbeispiele

2. Generalisierung der Konzepte, welche Mirror umsetzt.
--> Welche Konzepte nutzt Mirror?
--> Was ist daran nützlich?

3. Matchmaking
--> Moderne Matchmaking-Architekturen heute
--> Einordnung der Konzepte für das Matchmaking
--> HTTP Server / REST API / Dedicated Server 
--> Unterscheidung Hosting (Client Host / Dedicated Server Host)
--> Beispiel Implementierung Matchmaking System (Hide n seek)

4. Erstellung einer Doku für Matchmaking System
--> Doku der Implementierung 


\end{abstract}
\cleardoublepage