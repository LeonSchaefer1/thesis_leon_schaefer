% Wie schreibe ich das Abstract?
% Bei der Vorbereitung des Abstracts sollten Sie sich an folgenden grundlegenden Punkten orientieren:

%% Motivation des Textes:
%% worin liegt die Bedeutung der entsprechenden Forschung, warum sollte der längere Text gelesen werden?

%% Fragestellung:
%% welche Fragestellung(en) versucht der Text zu beantworten, was ist der Umfang der Forschung, was sind die zentralen Argumente und Behauptungen?

%% Methodologie:
%% welche Methoden/Zugänge nutzt der Autor/die Autorin, auf welche empirische Basis stützt sich der Text?

%% Ergebnisse:
%% zu welchen Ergebnissen kam die Forschung, was sind die zentralen Schlussfolgerungen des Textes?

%% Implikationen:
%% welche Schlussfolgerungen ergeben sich aus dem Text für die Forschung, was fügt der Text unserem Wissen über das Thema hinzu?
 
\begin{abstract}
Abstract:

Der Einstieg in die Multiplayer-Spieleeprogrammierung gestaltet sich für angehende Entwickler als schwierig. Ein erstes eigenes Projekt umzusetzen erfordert viel Einarbeitung und es gibt viele Einstiegshürden.

Das Ziel dieser Arbeit ist, eine Einstiegshilfe für die Entwicklung von Multiplayer-Spiele-Projekten zu geben. Hierzu wird eine abstrakte Vorlage konstruiert, welche auf möglichst viele Multiplayer Use-Cases anwendbar ist. 

Dazu wurden abstrahierte Konzepte erstellt, welche auf möglichst viele Spieleideen übertragbar sein sollen. Diese Konzepte bestehenden aus einzelnen Software-Komponenten, welche selbst noch keine konkrete Implementierungsvorgabe sind, sondern lediglich das Konzept der Komponente abstrakt beschreiben.

Als Proof of Concept wurden die beschriebenen Konzepte anhand einer Implementierung in einem Multiplayer-Spiel in der Unity Engine, mithilfe des Mirror Frameworks implementiert. 

\end{abstract}
\cleardoublepage