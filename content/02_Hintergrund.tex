% Alles was man wissen muss um das Konzept zu verstehen.

\chapter{Hintergrund}
\label{sec:hintergrund}

\section{Grundlagen des Networkings in Online-Videospielen}

Die Grundlagen des Networkings in Videospielen lassen sich in zwei große Bereiche kategorisieren. Die physische und die logische Plattform. 

Die physische Plattform setzt sich aus den physikalischen Komponenten zusammen, die zusammen eine Infrastruktur bilden. Hierzu zählt Hardware, welche in Rechenzentren eingesetzt wird, das lokale Endgerät wie z.B. ein Smartphone oder ein Personal Computer. Ebenfalls zählen Kabelleitungen und drahtlose Übertragungswege dazu. 

Online-Videospiele sind aus technischer Sicht auch nichts anderes als Anwendungen die miteinander kommunizieren. Die physischen Restriktionen wie Bandweite und Latenz können also ebenfalls auf den Kontext der Multiplayer-Spiele angewendet werden. Die Menge der Informationen, welche über ein Netzwerk versendet werden kann je nach Spieltyp sehr hoch skalieren, weshalb sich die Entwickler eines Multiplayer-Spiels intensiv mit der logischen Plattform beschäftigen müssen.

Die logische Plattform baut auf der physischen Plattform auf und nutzt ihre Ressourcen. Die logische Plattform eines Online-Multiplayer Spiels kann unterteilt werden in Kommunikation zwischen Clients und Server, Datenspeicherung und Transfer, und Kontrolle des Spielflusses. Auf diese drei Punkte wird in den folgenden Sektionen näher eingegangen.


\section{Verschiedene Arten des Hostings}

Die Kommunikation zwischen Clients kommt auf die Art des Hostings an, hierauf wird in der nächsten Sektion näher eingegangen.

\section{Datenspeicherung und Transfer}

Die Datenspeicherung erfolgt je nach Hosting auf unterschiedliche Art und Weise. Entweder

\section{Kontrolle des Spielflusses}

-> Interest Management


\section{Matchmaking}


// WEITERE IDEEN: Consistency and Responsiveness