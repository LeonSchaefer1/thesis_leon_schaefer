\chapter{Studienkonzept \& Zusammenfassung }
\label{sec:zusammenfassung}

\section{Studienkonzept}
\label{studienkonzept}

! STUDIENKONZEPT IST NOCH PRE-PRE-ALPHA (enthält Notizen)!

Wie bereits im Teil der Einführung erwähnt, hat diese Arbeit nicht den Anspruch zu beweisen, dass die Konzepte tatsächlich einen Mehrwert für eine Vielzahl an Projekten bieten. Um dies beweisen zu können, ist eine weitere Studie nötig. Im Rahmen dieser Arbeit wurde ein Studienkonzept erarbeitet, welches zukünftig für andere Arbeiten genutzt werden kann.

Das Studienkonzept gliedert sich in:

Repräsentative Umfragen (Umsetzung mit Likert-Skala):
Diese beinhaltet die Befragung von mehreren Hundert Personen aus unterschiedlichen Ländern sowie unterschiedlichen Online-Genres (MMORPG, First Person Shooter, Mobile-Games, etc.).

Folgende Fragestellungen müssen untersucht werden:

Welche Probleme haben angehende Entwickler für Online-Multiplayer Spiele?
Lassen sich diese Probleme auf die in dieser Arbeit beschriebenen Konzepte übersetzen?
Bringen die hier beschriebenen Konzepte einen Vorteil in der Praxis?

Zunächst müssen Frage - und Problemstellungen, welche als Annahme getroffen wurden, in Fragen umgewandelt werden. Diese Frage - und Problemstellungen beziehen sich auf konkrete Themen der Multiplayer-Spielenentwicklung.

Fragen an junge, angehende Online-Spieleentwickler, welche ein eigenes Projekt umgesetzt haben bzw. welches sich aktuell in Umsetzung befindet:

Problemstellung/Annahme: Es gibt inzwischen viele Frameworks und Engines zur Entwicklung von Online-Multiplayerspielen \cite{MFatihMAR.2021}. Die Entscheidung eine passende Technologie auszuwählen ist mühsam.

Abgeleitete Frage: Wie leicht kamen sie zurecht bei der Auswahl eines Frameworks und/oder eine Game-Engine für ihr erstes Online-Multiplayer Projekt?

Auswahlmöglichkeiten: Auswahl fiel mir sehr leicht, Auswahl fiel mir leicht, Auswahl fiel mir schwer, Auswahl fiel mir sehr schwer

Wie schwierig empfanden sie die Handhabung von 'Allgemeine Beschreibung Konzept 1' in ihrem ersten Projekt?

Wie schwierig empfanden sie die Handhabung von 'Allgemeine Beschreibung Konzept 2' in ihrem ersten Projekt?

...weitere Fragen...

Wie schwierig empfanden sie die Handhabung von 'Allgemeine Beschreibung Konzept N' in ihrem ersten Projekt?

--> Umfrage veröffentlichen

Die Umfrage soll per Email an ausgewählte Personen an Hochschulen und Unternehmen verschickt werden. Außerdem kann die Umfrage in einschlägige Entwicker-Foren gepostet werden. Beispiele hierfür sind: forum.unity.com , forums.unrealengine.com, www.gamedev.net/forums/forum/8-networking-and-multiplayer.

Die Umfrage kann anonymm, oder mit Personenbezug für eine bessere Auswertung der Ergebnisse erfolgen. Ebenso wäre es denkbar getrennte Umfragen für eine Gruppe an ausgewählten Einzelpersonen sowie die öffentliche Entwickler-Community im Internet zu erstellen.

--> Parallel Versuch starten: 2 Gruppen von Anfänger Entwicklern eine Aufgabe geben (Multiplayer-Spiel bauen), eine bekommt die Konzepte, die andere nicht.

- Control Experiment -> Mehr Beschreiben: Was für Variablen gibt es?
-> Anforderungen an ein Spiel (Wurde alles erfüllt)
-> Methodisches Vorgehen

Mithilfe eines Experiments (oder mehrerer Experimente) soll gezeigt werden, dass Einstiegs-Entwickler in der Multiplayer-Spieleprogrammierung einen klaren Vorteil haben, wenn sie das technische Design eines Neuprojekts nach diesen Konzepten ausrichten.

Hierfür werden 2 Gruppen aus jeweils maximal 3 Personen bestimmt, welche beide ein identisches, rudimentäres Online-Spiel entwickeln sollen. Die eine Gruppe bekommt die Konzepte als Orientierung zur Verfügung, die andere nicht.

Die Anforderungen an die Probanden sind:
--> Sollten die Probanden aus Studenten bestehen, so müssen alle Probanden im gleichen Semester sein.
--> Die Probanden müssen gleich viel (am besten keine, oder wenig) Erfahrung im Bereich der Multiplayer-Spieleentwicklung haben.

Die Anforderungen an das Experiment sind:
--> Es muss die Zeit gemessen werden, wie viel Stunden jede Gruppe zur Fertigstellung des vorgegebenen Projekts benötigt werden.
--> Alle Probanden müssen Zugang zu ähnlicher Hardware besitzen.
--> Die Gruppen teilen die Aufgaben unter sich auf, und suchen sich selbstständig Frameworks und Engines für das Projekt aus.
--> Im Anschluss müssen beide Gruppen die oben erarbeitete Umfrage ausfüllen.


--> Ergebnisse der Umfrage auswerten, passen die Ergebnisse zu der hier beschriebenen Problemstellung?

Die Ergebnisse der Umfrage an ausgewählte Personen und an die Entwickler Community werden ausgewertet. Kann man einen Trend erkennen? Können die Problemstellungen / Thesen validiert werden? 
(Konkret: Sind die Annahmen dieser Arbeit korrekt? Haben Anfänger-Entwickler tatsächliche Schwierigkeiten mit den Punkten X, Y, Z ..?)

--> Auswertung des Experiments / der Experimente: Hat die Gruppe, welche sich an die Konzepte gehalten hat einen messbaren Vorteil gehabt? (Geschwindigkeit, Anfangsschwierigkeiten)

Die Daten aus dem Experiment (Zeitmessung, Umfrageergebnisse) sollen nun wie folgt ausgewertet werden:

--> Ist die Gruppe, welche das Projekt mit den Konzepten umgesetzt hat, schneller zum Ziel gekommen, als die Gruppe ohne Hilfestellung der Konzepte?

--> Hatte die Gruppe, welche das Projekt mit den Konzepten umgesetzt hat, weniger Start-Schwierigkeiten bei der Umsetzung des Projekts, als die Gruppe ohne Hilfestellung der Konzepte?


Weitere Punkte, die in Studienkonzept noch fehlen:
--> Methodik für semi-strukturiertes Experteninterviews beschreiben

\section{Weitere Ansätze}

--> Weitere Konzepte, die mir zu konkret waren (Player Movement Controller, etc.)

\section{Auswertung \& Fazit}

--> Bezug nehmen auf Forschungsfragen!
--> Kein Beweis für die 'Allgemeine Gültigkeit'
--> Für Prototyp hat es geholfen

\section{Nächste Schritte \& Ausblick}

--> Finale Erkenntnisse müssen durch Studienkonzept erarbeitet werden