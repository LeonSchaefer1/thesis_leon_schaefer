\chapter{Studienkonzept \& Zusammenfassung }
\label{sec:zusammenfassung}

In diesem Kapitel wird zunächst ein Studienkonzept erläutert, welches weitere Forschung ermöglicht. Es folgt ein Fazit und Ausblick.

\section{Studienkonzept}
\label{studienkonzept}

Um zu beweisen, dass die vorgestellten Konzepte einen tatsächlichen Mehrwert für die Entwicklung von Multiplayer-Spielen bietet, bedarf es einer Studie, welche im folgenden Studienkonzept beschrieben wird.

\textbf{Quantitative Umfrage:}

Diese beinhaltet die Befragung von mehreren Hundert Personen aus unterschiedlichen Ländern und Genres. Sie soll mithilfe der Likert-Skala \cite{Wikipedia.2022d} einige persönliche Einstellungen der befragten Personen messen. Diese haben starken Bezug zu den Konzepten, welche in dieser Arbeit beschrieben wurden. Um die Einstellungen der befragten Personen zu den einzelnen Entwicklungs-Problemen (Items) zu erfassen, werden stets die gleichen Antwortmöglichkeiten (Grade) angeboten:

trifft zu (1), trifft eher zu (2), teils-teils (3), trifft eher nicht zu (4), trifft nicht zu (5)

Die Umfrage sollte folgende Items beinhalten:

Die Entscheidung eine passende Technologie für mein erstes Multiplayer-Spiele auszuwählen war mühsam.

Die Entwicklung des Matchmaking Systems bereitete mir Probleme.

Die Umsetzung des Server-Runners bereitete mir Probleme.

Die Implementierung der Benutzeroberfläche bereitete mir Probleme.

Die Realisierung der Verarbeitung von Netzwerkereignissen bereitete mir Probleme.

Die Umsetzung der Szenen-Verwaltung bereitete mir Probleme.

Die Entwicklung des clientseitigen Verbindungsaufbaus zu einem Spiel-Server bereitete mir Probleme.

Die Entwicklung der serverseitigen Verwaltung aller Spieler bereitete mir Probleme.

Die Umsetzung der Vorbereitung von Spiel-Szenen bereitete mir Probleme.

Die Realisierung der Verwaltung von Spielfortschritt bereitete mir Probleme.

Die Implementierung des Spawnings von Spielobjekten bereitete mir Probleme.

Die Entwicklung der Sichtbarkeits-Verwaltung von Objekten bereitete mir Probleme.

Die Umfrage soll per E-Mail an ausgewählte Personen an Hochschulen und Unternehmen verschickt werden. Außerdem kann die Umfrage in einschlägige Entwickler-Foren gepostet werden. Beispiele hierfür sind: \href{www.forum.unity.com}{www.forum.unity.com}, \href{www.forums.unrealengine.com}{www.forums.unrealengine.com} und \href{www.gamedev.net/forums/forum/8-networking-and-multiplayer}{www.gamedev.net/forums/forum/8-networking-and-multiplayer.}. Wichtig ist, dass lediglich Personen an der Umfrage teilnehmen, welche aktuell ihr erstes Multiplayer-Spiel entwickeln oder bereits entwickelt haben. Die Teilnehmer sollten darauf hingewiesen werden, dass falls sie noch keine Berührungspunkte mit einzelnen Punkten haben, keine Antwort auf die entsprechende Fragestellung geben sollen.

Die Umfrage kann anonym, oder mit Personenbezug für eine bessere Auswertung der Ergebnisse erfolgen. Ebenso wäre es denkbar, getrennte Umfragen für eine Gruppe an ausgewählten Einzelpersonen sowie die öffentliche Entwickler-Community im Internet zu erstellen.

\textbf{Feldversuch:}

Mithilfe dieses Feldversuchs soll gezeigt werden, dass unerfahrene Entwickler einen klaren Vorteil haben, wenn sie das technische Design eines Neuprojekts nach diesen Konzepten ausrichten. 

2 Gruppen von unerfahrenen Entwicklern wird die Aufgabe gegeben, ein Multiplayer-Spiel zu entwickeln. Eine Gruppe bekommt die Konzepte \hyperref[sec:konzepte]{aus dem Kapitel 'Konzepte'} als Hilfestellung, die andere nicht.

Die generellen Anforderungen an die Feldstudie sind:

- Sollten die Probanden aus Studenten bestehen, so müssen sich alle im gleichen Fachsemester befinden.
- Die Probanden sollten gleich viel Erfahrung im Bereich der Multiplayer Spieleentwicklung haben.
- Alle Probanden müssen Zugang zu ähnlicher Hardware besitzen.
- Die Gruppen teilen die Aufgaben unter sich auf, und suchen sich selbstständig Frameworks und Engines für einen Prototyp aus.
- Im Anschluss müssen beide Gruppen die oben erarbeitete Umfrage ausfüllen.
- Das Experiment ist zeitlich begrenzt. Ein Ergebnis muss innerhalb von 7 Tagen eingereicht werden.

Die Anforderungen an das Spiel sind:

- Das Spiel soll ein simpler Multiplayer-Ableger des Spiels 'Snake' werden. Es müssen lediglich die Grundregeln des Spiels umgesetzt werden. Diese sind hier: \cite{.22.02.2022} beschrieben.
- Es soll für die Spieler möglich sein, sich über eine Lobby zu treffen, und von dort aus ins Spiel zu starten.
- Die Grundarchitektur soll dem Client - Server Modell entsprechen.

Während des Experiments sollen folgende Variablen dokumentiert werden:

- Wurden alle Anforderungen an das Spiel erfüllt?
- In welcher Zeit hat welches Team den Prototypen entwickelt?
- Wie sind beide Teams methodisch vorgegangen?

Je mehr Feldstudien dieser Art durchgeführt werden, desto genauer kann später eine finale Aussage getroffen werden.

\textbf{Auswertung der quantitativen Umfrage:}

Die Ergebnisse der Umfrage an ausgewählte Personen und an die Entwickler Community werden ausgewertet. Aus den Ergebnissen sollten Antworten auf folgende Fragestellungen abgeleitet werden können:

Ist ein Trend zu erkennen, dass angehende Spieleentwickler Probleme mit den aufgeführten Implementierungsdetails hatten?

Bei welchen Implementierungsdetails hatten Entwickler am meisten Probleme?

\textbf{Auswertung der Feldstudie:}

Hat die Gruppe, welche sich an die Konzepte gehalten hat, einen messbaren Vorteil gehabt? Konkret sollten die oben beschriebenen Variablen ausgewertet werden:

Ist die Gruppe, welche den Prototypen mit den Konzepten umgesetzt hat, schneller zum Ziel gekommen, als die Gruppe ohne Hilfestellung?

Wurden bei beiden Gruppen alle Anforderungen umsetzt?

Hatte das Team, welches den Prototypen mithilfe der Konzepte umgesetzt hat weniger Probleme die Arbeitslast innerhalb des Teams aufzuteilen?

Anschließend wird die Umfrage betrachtet, welche die Teilnehmer ausgefüllt haben. Aus dieser kann abgeleitet werden, ob die Entwickler der Gruppe, welche eine Hilfestellung erhalten haben, einen spürbaren Vorteil bei der Entwicklung des Prototyps hatten.

Zum Schluss können nun die quantitative Umfrage sowie die Umfrage, welche aus dem Ergebnis der Feldstudie resultiert, miteinander verglichen werden. Sollten die Personen, welche die Feldstudie innerhalb der Gruppe ohne Hilfestellung durchgeführt haben, eine ähnliche Einstellung zu den Punkten der Umfrage haben, wie die Personen, welche an der quantitativen Umfrage teilgenommen haben, so ist davon auszugehen, dass beide Personengruppen eine ähnliche Erfahrung bei der Entwicklung von Multiplayer-Spielen mitbringen, und diese Personengruppen somit als Vergleichbar gelten.

Sollte die Gruppe, welche die Konzepte als Hilfestellung genutzt hat, die Umfrage mit einem anderen Ergebnis abschließen, so kann dies ein Hinweis sein, dass die Konzepte, welche in dieser Arbeit beschrieben sind, eine Einstiegshilfe bei der Entwicklung von Multiplayer-Spielen bietet.

Um dies final zu beweisen, müsste die Feldstudie mehrmals durchgeführt werden, und die Anforderungen an das zu entwickelnde Spiel müssen bei jeder neuen Feldstudie geändert werden. Mehr Variation würde dann geschaffen, wenn sich Genre, Matchmaking, Art des Hostings und Spielkonzept bei jeder neu durchgeführten Feldstudie ändert.


\section{Fazit}

Die in dieser Arbeit gestellten Forschungsfragen \hyperref[f1]{F1} und \hyperref[f2]{F2} konnten teilweise beantwortet werden.
\hyperref[f1]{F1:} Durch die Aufstellung der Konzepte, und anschließenden Umsetzung eines Prototyps kann nicht final beurteilt werden, ob die Probleme für eine Vielzahl an angehenden Entwicklern vereinfacht wurden. Allerdings war die Entwicklung des in dieser Arbeit beschriebenen Prototyps durch die Einhaltung der Konzepte deutlich vereinfacht. Dies könnte ein Hinweis darauf sein, dass Entwickler mit anderen Spielideen, welche ähnliche Voraussetzungen besitzen wie die des Prototyps, welcher im Rahmen dieser Arbeit entwickelt wurde, einen Vorteil aus den Konzepten ziehen könnten.

Für Spielideen, welche aus einem anderen Genre kommen, eine andere Form des Matchmakings oder Art des Hostings unterstützen, kann jedoch ohne Ausführung einer weitergehenden Studie keine Aussage getroffen werden.

\hyperref[f2]{F2:} Ein Studienkonzept könnte so aussehen, wie es in der Sektion \hyperref[studienkonzept]{Studienkonzept} beschrieben ist. Bestimmte Details können je nach konkreter weitergehender Forschungsfrage abgeändert oder ergänzt werden. Grundsätzlich führt eine korrekte Durchführung des Studienkonzepts zur Beantwortung der Forschungsfrage:
'Bringen die in dieser Arbeit beschriebenen Konzepte einen Mehrwert bei der Entwicklung eines Multiplayer-Spiels für angehende Entwickler?'

\section{Ausblick}

Abschließend ist festzuhalten, dass diese Arbeit einen Ansatz bietet, welcher zur Erleichterung des Einstiegs in die Entwicklung von Multiplayer-Spielen führen könnte. Um zu beweisen, dass die Konzepte praxistauglich und für eine Vielzahl an unterschiedlichen Projekten einsetzbar sind, muss jedoch eine weitergehende Studie durchgeführt werden, welche sich an dem in dieser Arbeit beschriebenen \hyperref[studienkonzept]{Studienkonzept} orientiert. Aus den Ergebnissen der Studie können weitere Schlussfolgerungen gezogen werden, die eine abschließende Bewertung dieser Arbeit ermöglichen.