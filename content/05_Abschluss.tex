\chapter{Studienkonzept \& Zusammenfassung }
\label{sec:zusammenfassung}

\section{Studienkonzept}
\label{studienkonzept}

Um zu beweisen, dass die vorgestellten Konzepte einen tatsächlichen Mehrwert für die Entwicklung von Multiplayer-Spielen bietet, bedarf es einer Studie, welche im folgenden Studienkonzept beschrieben wird.

\textbf{Repräsentative Umfragen:}

Diese beinhaltet die Befragung von mehreren Hundert Personen aus unterschiedlichen Ländern sowie unterschiedlichen Genres. Sie soll mithilfe der Likert-Skala \cite{Wikipedia.2022d} einige persönliche Einstellungen der befragten Personen messen. Diese haben starken Bezug zu den Konzepten, welche in dieser Arbeit beschrieben wurden. Um die Einstellungen der befragten Personen zu den einzelnen Entwicklungs-Problemen (Items) zu erfassen, werden stets die gleichen Antwortmöglichkeiten (Grade) angeboten:

trifft zu (1), trifft eher zu (2), teils-teils (3), trifft eher nicht zu (4), trifft nicht zu (5)

Die Umfrage sollte folgende Items beinhalten:

Die Entscheidung eine passende Technologie für mein erstes Mutliplayer-Spieleprojekt auszuwählen war mühsam.

Die Implementierung des Matchmaking Systems bereitete mir Probleme.

Die Implementierung des Server-Runners bereitete mir Probleme.

Die Implementierung der Benutzeroberfläche bereitete mir Probleme.

Die Implementierung der Verarbeitung von Netzwerkereignissen bereitete mir Probleme.

Die Implementierung der Szenen-Verwaltung bereitete mir Probleme.

Die Implementierung des clientseitigen Verbindungsaufbaus zu einem Spiel-Server bereitete mir Probleme.

Die Implementierung der serverseitigen Verwaltung aller Spieler bereitete mir Probleme.

Die Implementierung der Vorbereitung von Spiel-Szenen bereitete mir Probleme.

Die Implementierung der Verwaltung von Spielfortschritt bereitete mir Probleme.

Die Implementierung des Spawnings von Spielobjekten bereitete mir Probleme.

Die Implementierung der Sichtbarkeits-Verwaltung von Objekten bereitete mir Probleme.

Die Umfrage soll per Email an ausgewählte Personen an Hochschulen und Unternehmen verschickt werden. Außerdem kann die Umfrage in einschlägige Entwicker-Foren gepostet werden. Beispiele hierfür sind: \href{www.forum.unity.com}{www.forum.unity.com}, \href{www.forums.unrealengine.com}{www.forums.unrealengine.com} und \href{www.gamedev.net/forums/forum/8-networking-and-multiplayer}{www.gamedev.net/forums/forum/8-networking-and-multiplayer.}

Die Umfrage kann anonymm, oder mit Personenbezug für eine bessere Auswertung der Ergebnisse erfolgen. Ebenso wäre es denkbar getrennte Umfragen für eine Gruppe an ausgewählten Einzelpersonen sowie die öffentliche Entwickler-Community im Internet zu erstellen.

\textbf{Feldversuch:}

Mithilfe dieses Feldversuchs soll gezeigt werden, dass unerfahrene Entwickler in der Multiplayer-Spieleprogrammierung einen klaren Vorteil haben, wenn sie das technische Design eines Neuprojekts nach diesen Konzepten ausrichten. 

2 Gruppen von unerfahrenen Entwicklern wird die Aufgabe gegeben ein Multiplayer-Spiel zu entwickeln. Eine Gruppe bekommt die Konzepte \hyperref[sec:konzepte]{aus dem Kapitel 'Konzepte'} als Hilfestellung, die andere nicht.

Die generellen Anforderungen an die Feldstudie sind:

- Sollten die Probanden aus Studenten bestehen, so müssen sich alle im gleichen Fachsemester befinden.
- Die Probanden sollten gleich viel Erfahrung im Bereich der Multiplayer Spieleentwicklung haben.
- Alle Probanden müssen Zugang zu ähnlicher Hardware besitzen.
- Die Gruppen teilen die Aufgaben unter sich auf, und suchen sich selbstständig Frameworks und Engines für das Projekt aus.
- Im Anschluss müssen beide Gruppen die oben erarbeitete Umfrage ausfüllen.

Die Anforderungen an das Spiel sind:

- Das Spiel soll ein simpler Mehrspieler-Ableger des Spiels 'Snake' werden. Es müssen lediglich die Grundregeln des Spiels umgesetzt werden. Diese sind hier: \cite{.22.02.2022} beschrieben.
- Es soll für die Spieler möglich sein, sich über eine Lobby zu treffen, und von dort aus ins Spiel zu starten.
- Die Grundarchitektur soll dem Client - Server Modell entsprechen.

Während des Experiments sollen folgende Variablen dokumentiert werden:

- Wurden alle Anforderungen an das Spiel erfüllt?
- In welcher Zeit hat welches Team das Projekt abgeschlossen?
- Wie sind beide Teams methodisch vorgegangen?

\textbf{Auswertung:}

-> Auswertung Umfrage
-> Auswertung Feldversuch

Die Ergebnisse der Umfrage an ausgewählte Personen und an die Entwickler Community werden ausgewertet. Kann man einen Trend erkennen? Können die Problemstellungen / Thesen validiert werden? 
(Konkret: Sind die Annahmen dieser Arbeit korrekt? Haben Anfänger-Entwickler tatsächliche Schwierigkeiten mit den Punkten X, Y, Z ..?)

--> Auswertung des Experiments / der Experimente: Hat die Gruppe, welche sich an die Konzepte gehalten hat einen messbaren Vorteil gehabt? (Geschwindigkeit, Anfangsschwierigkeiten)

Die Daten aus dem Experiment (Zeitmessung, Umfrageergebnisse) sollen nun wie folgt ausgewertet werden:

--> Ist die Gruppe, welche das Projekt mit den Konzepten umgesetzt hat, schneller zum Ziel gekommen, als die Gruppe ohne Hilfestellung der Konzepte?

--> Hatte die Gruppe, welche das Projekt mit den Konzepten umgesetzt hat, weniger Start-Schwierigkeiten bei der Umsetzung des Projekts, als die Gruppe ohne Hilfestellung der Konzepte?


Weitere Punkte, die in Studienkonzept noch fehlen:
--> Methodik für semi-strukturiertes Experteninterviews beschreiben

\section{Weitere Ansätze}

--> Weitere Konzepte, die mir zu konkret waren (Player Movement Controller, etc.)

\section{Auswertung \& Fazit}

--> Bezug nehmen auf Forschungsfragen!
--> Kein Beweis für die 'Allgemeine Gültigkeit'
--> Für Prototyp hat es geholfen

\section{Nächste Schritte \& Ausblick}

--> Finale Erkenntnisse müssen durch Studienkonzept erarbeitet werden