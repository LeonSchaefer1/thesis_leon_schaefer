\chapter{Realisierung}
\label{sec:realisierung}

In diesem Kapitel werden die im vorherigen Kapitel beschriebenen abstrakten Konzepte anhand von einem Prototypen umgesetzt. Die konkrete Logik wird bei jedem implementierten Konzept anhand von Codebeispielen erläutert.

\section{Eingesetzte Technologien}

Für die Entwicklung des Prototyps kamen folgende Technologien zum Einsatz:

- Als Game-Engine wurde Unity\cite{Technologies.03.02.2022} benutzt. Die Entscheidung hierfür begründet sich durch die gute Dokumentation und den Long-Time-Support der verschiedenen Unity Versionen. Außerdem gibt es eine große Anzahl an Benutzer. Unity existiert bereits seit dem Jahr 2005 \cite{Wikipedia.2022c}, und hat sich seitdem in Entwickler-Kreisen einen nachhaltigen Ruf erarbeitet.

- C\# kommt als Programmiersprache zum Einsatz. Unity hat eine sehr gute Unterstützung hierfür.

- Zusätzlich zu Unity wurde Mirror\cite{.03.02.2022} als weiteres Framework benutzt. Mirror ist eine kostenloses Open Source Networking Bibliothek, welche die Entwicklung von Netzwerk-Code deutlich erleichtert.

\section{Spielidee des Prototyps}
\label{Spielidee}

// TODO Add Content

--> Hide and Seek kurz erklären

--> Join Random Lobby / Zugangscode erklären

\section{Architektur des Prototyps}
\label{Architektur}

// TODO Add Content

--> Miro-Diagramm mit verwendeten Konzepten

--> Spielflussdiagramm (inkl. Szenen-Erklärung (GameScene = Spielszene, MenuScene = Menüszene))

MenuScene:

GameScene:

\section{Implementierung: API für Matchmaking \& Server Runner}

// TODO: Add Content

\section{Implementierung: Client UI \& Visual Controller}

Der Client UI \& Visual Controller spielt im Prototyp eine wesentliche Rolle. Er kommt in der GameScene als Singleton \cite{M.Gatrell.2009} zum Einsatz, wo er alle nötigen Methoden implementiert, um das lokale UI des Spielers manipulieren zu können. Hier sind nun 3 Beispiele aufgeführt:

Die folgenden "Update" Methode ist eine Unity bereitgestellte Schnittstelle. Der Rumpf der Update Methode ist nie vorgegeben, und muss vom Entwickler selbst implementiert werden. Unity führt diese Methode jedes "Frame" (Einzelbild) \cite{Wikipedia.2021j} aus. Dies ermöglicht eine nahezu vollständige Echtzeitabfrage der Inputs (Tastatur / Maus / andere Eingangsgeräte) des Spielers.

Im unten wird dauerhaft abgefragt, ob ein Spieler die Taste "E" oder "Q" auf seiner Tastatur drückt. Wenn er das tut, und ebenfalls ein paar andere Randbedingungen erfüllt sind, so wird ein "Linksklick" auf den im UI des Spielers sichtbaren Button simuliert.

\begin{lstlisting}[caption= InGameUiControllerScript.cs Update Method]
private void Update()
{
	if (Input.GetKeyDown(KeyCode.E) && hotkeyImage != null && hotkeyImage.sprite == HOTKEY_KEYBOARD_E && interactButton.GetComponent<Button>().interactable )
	{
		interactButton.GetComponent<Button>().onClick.Invoke();
	}
	if (Input.GetKeyDown(KeyCode.Q) && lightButton.GetComponent<Button>().interactable)
	{
		lightButton.GetComponent<Button>().onClick.Invoke();
	}
}
\end{lstlisting}

Doch was heißt das genau? Im folgenden Code wird ein Event definiert, welches durch andere Klassen "aboniert" werden kann. Wird dann das Event durch .Invoke() ausgelöst, so werden alle Funktionen in externen Skripten ausgeführt, die dieses Event aboniert haben. Hier zunächst die Definition des Events im InGameUiControllerScript und die Funktion "clickInteractButton", die ausgeführt wird, wenn beispiese wie im obrigen Code die Taste "E" auf der Tastatur gedrückt wird.

\begin{lstlisting}[caption= InGameUiControllerScript.cs OnInteractButtonClick Event]
public event Action OnInteractButtonClick;	

public void clickInteractButton()
{
	OnInteractButtonClick?.Invoke();
}
\end{lstlisting}

Um den gesamten Aufrufstapel\cite{Wikipedia.2021k} nachvollziehen zu können, folgt nun ein Beispiel aus einem anderen Script.
Die Klasse "HiderScript.cs" aboniert nun in seiner Methode "RpcEnableHideButton()" das Event des InGameUiControllerScript Singletons "OnInteractButtonClick" mit seiner eigenen Methode "OnHiderHideButtonClick()". Die Methode "RpcEnableHideButton()" wird dann ausgeführt, wenn der Server einem Spieler mit der Roller "Hider" erlaubt, ein Versteck betreten zu dürfen.

\begin{lstlisting}[caption= HiderScript.cs Subscribe to InGameUiControllerScript Event]
[TargetRpc]
private void RpcEnableHideButton()
{
	InGameUiControllerScript.singleton.resetInteractButtonClickEvents();
	InGameUiControllerScript.singleton.OnInteractButtonClick += OnHiderHideButtonClick;
	InGameUiControllerScript.singleton.setInteractButtonEnabled(true);
	InGameUiControllerScript.singleton.setInteractButtonTextAndHotkeyImage("Hide");
}
\end{lstlisting}

Sollte der Hider nun auf seinen "Interact"-Button drücken, so wird die folgende Methode "OnHiderHideButtonClick()" aufgerufen. Diese ruft eine weitere Methode "CmdHideHider()" auf, welche einen Remote Procedure Call\cite{.05.02.2022} zum Server darstellt.

\begin{lstlisting}[caption= HiderScript.csOnHiderHideButtonClick() Method]

[Client]
private void OnHiderHideButtonClick()
{
	if (enabled)
	{
		CmdHideHider();
	}
}

\end{lstlisting}

Der Server führt nun innerhalb des Methodenrumpfs von "CmdHideHider()" einige Überprüfungen durch, ändert server-interne Variablen und sorgt somit dafür, dass alle Spieler darüber bescheid wissen, dass sich der Spieler nun in einem Versteck befindet.

\begin{lstlisting}[caption= HiderScript.cs Subscribe to InGameUiControllerScript Event]
	
	[Command]
	private void CmdHideHider()
	{
		if(vicinityScript.getHideoutObjectScript() != null)
		{
			HideableObjectScript curHideoutScript = vicinityScript.getHideoutObjectScript();
			if (!vicinityScript.getHideoutObjectScript().getIsTaken())
			{
				GetComponent<PlayerBaseScript>().setCurHideout(curHideoutScript.gameObject);
				curHideoutScript.setIsTaken(true);
				curHideoutScript.setCurrentHider(gameObject);
				isHiding = true;
				GetComponent<PlayerBaseScript>().playerLightEnabled = false;
				GameNetworkManager.FindObject(gameObject, "FeetTrigger").SetActive(!isHiding);
				RpcNotifyHideState(true, GameNetworkManager.singleton.gameRules[4].playerSpeed);
				RpcEnableLeaveHideoutButton();
				GetComponent<PlayerBaseScript>().TargetEnableLightButton(false);
			}
			else
			{
				RpcDisableInteractButton();
			}
		}
	}	
\end{lstlisting}

Die nächste Beispielmethode implementiert die Logik, den Interact-Button selbst aktiv- und inaktiv zu setzen. Der Interact Button selbst verfügt ebenfalls über ein "HotkeyImage", welches einem visuellen Indikator gleich kommt, der die aktuell zu drückende Taste auf der Tastatur darstellt. Dieser Indikator wird bei jedem Aufruf von setInteractButtonEnabled ebenfalls an - oder ausgeschaltet.

\begin{lstlisting}[caption= InGameUiControllerScript.cs setInteractButtonEnabled]
public void setInteractButtonEnabled(bool enabled)
	{
		interactButton.GetComponent<Button>().interactable = enabled;
		if(hotkeyImage != null)
		hotkeyImage.gameObject.SetActive(enabled);
	}
\end{lstlisting}

Das letzte Beispiel ist eine Implementung eines simplen Flip-Mechanismus bei Betätigen des Light-Buttons (An- und ausschalten der Taschenlampe eines Spielers). Diese Logik tauscht lediglich das Bild (Sprite) aus, welches auf dem Button liegt.

\begin{lstlisting}[caption= InGameUiControllerScript.cs flipLightButtonImage]
public void flipLightButtonImage()
{
	if (!lightOn)
	{
		lightButton.GetComponent<Image>().sprite = lightOnSprite;
	}
	else
	{
		lightButton.GetComponent<Image>().sprite = lightOffSprite;
	}
	lightOn = !lightOn;
}
\end{lstlisting}


\section{Implementierung: Server Network Manager}

In der Implementierung des Prototyps ist der Server Network Manager ein Teil der GameNetworkManager Klasse. Diese implementiert sowohl Client- als auch Serverseitige Netzwerkfunktionen. Die nicht vorhandene Separierung von Client-und Serverfunktionen ist dem Mirror Framework geschuldet. Die Klasse GameNetworkManager ist ebenfalls ein Singleton und erbt die Klasse NetworkManager, welche von Mirror bereitgestellt wird. Durch die Vererbung ist es möglich, virtuelle Methoden\cite{Billwagner.08.02.2022} der NetworkManager Klasse zu überschreiben, Netzwerkseitige Events abzufangen und mit eigener Logik auf diese zu reagieren. 

Im folgenden Beispiel wird die virtuelle Methode OnServerConnect() überschrieben, um zu überprüfen, ob ein Spieler einer Spiel-Session beitreten darf. Sollte sich eine Spielsession bereits in der Spiel-Szene befinden, so wird eine neue Message erzeugt, welche an den Client geschickt wird, welcher aktuell versucht sich zu verbinden. Das struct "Message" nutzt ein von Unity bereitgestelltes Interface "NetworkMessage", welches eine leichtgewichtige Art und Weise der Netzwerk-Kommunikation ermöglicht. Das gleiche Schema wird im zweiten "if" Block benutzt, hier wird überprüft, ob die maximale Anzahl an Spielern innerhalb einer Lobby bereits erreicht ist.

\begin{lstlisting}[caption= GameNetworkManager.cs OnServerConnect() und Message struct]
public struct Message : NetworkMessage
{
	public string message;
	public MessageType messageType;
}
	
public override void OnServerConnect(NetworkConnection conn)
{
	// If the game server is already in gameScene, forbid access to game
	if (singleton.getNetworkSceneName() == "Assets/Scenes/GameScene.unity")
	{
		Message msg = new Message()
		{
			message = "Game has already started",
			messageType = MessageType.gameAlreadyStartedError
		};
		conn.Send(msg);
		conn.Disconnect();
	}
	
	// Forbid access to a lobby, if the lobby is already at maxPlayers (configurable inside GameNetworkManager Component)
	if(lobbyPlayers.Count == maxPlayers)
	{
		Message msg = new Message()
		{
			message = "Maximum Players reached for this lobby",
			messageType = MessageType.tooManyPlayersError
		};
		conn.Send(msg);
		conn.Disconnect();
	}
}
\end{lstlisting}

Als Gegenstück zu OnServerConnect() gibt es ebenfalls die virtuelle Methode OnServerDisconnect(), welche dann aufgerufen wird, wenn eine Spieler den Server gewollt, oder ungewollt (bspw. durch Netzwerkabbruch) verlässt. Im Prototyp wurde diese Methode ebenfalls überschrieben und im GameNetworkManager selbst implementiert. 

Zunächst wird überprüft, ob die Anzahl aller Spieler, welche sich aktuell auf dem Server befinden nach dem Beenden der aktuell abgefangenen und beendeten Client-Verbindung 0 beträgt. Falls ja, wird der Server-Prozess beendet. 

Falls sich noch Spieler auf dem Server-Prozess befinden werden je nach Spielszene verschiedene Szenarien durchlaufen. Falls die Spielszene aktuell läuft, so wird überprüft ob die Anzahl an Spielern mit der Rolle "Hider" oder "Seeker" aktuell 0 beträgt. Falls eine der beiden Fälle zutrifft, so wird das Spiel zu Gunsten des konkurrierenden Teams beendet.

Sollte sich die Spiel-Session aktuell in der Menü-Szene befinden, so werden Lobby-Relevante Variablen aktualisiert und zunächst überprüft, ob der Spieler, welche grade den Server verlassen hat, den Leader-Status hatte. Sollte dieser Fall zutreffen, wird der Leader-Status weiter gegeben. Zuletzt werden alle Spieler über den neuen Zustand der Lobby informiert.

\begin{lstlisting}[caption= GameNetworkManager.cs OnServerDisconnect()]
public override void OnServerDisconnect(NetworkConnection conn)
{
	if ((NetworkServer.connections.Count == 0)
	{
		Debug.Log("Server shutdown because all Players left");
		Application.Quit();
	}

	if (SceneManager.GetActiveScene().path == gameScene)
	{
		// if Seeker disconnected and no seeker left, let hiders win
		if(inGameProgressManager.getTotalSeekerCount() == 0)
		{
			inGameProgressManager.notifyHidersWinEvent();
		}
		
		// If Hider disconnected and no hiders left, let seekers win
		if(inGameProgressManager.getTotalHidersCount() == 0)
		{
			inGameProgressManager.notifySeekerWinEvent();
		}
	}
	else if(conn.identity != null)
	{
		lobbyPlayers.Remove(conn);
		LobbyPlayer curPlayerScript = conn.identity.GetComponent<LobbyPlayer>();
			
		if (curPlayerScript.isLeader)
		{
			// Pass Leader State to next Player
			var newLeader = lobbyPlayers.First().Value;
			newLeader.isLeader = true;
			leader = newLeader;
			leader.TargetNotifyLobbyReady(isLobbyReady);
		}
			
		var index = 0;
		// Pass the index
		foreach (var lobbyPlayer in lobbyPlayers.Values)
		{
			lobbyPlayer.index = index;
			index++;
		}
	}
		
	UpdateLobbyReadyState();
}
\end{lstlisting}

Weitere virtuelle Methoden, welche der Prototyp implementiert hat sind:

OnServerChangeScene() - Diese Methode wird ausgeführt, sobald ein Szenenwechsel bevorsteht, aber noch nicht vollzogen ist.

OnServerSceneChanged() - Diese Methode wird ausgeführt, sobald ein Szenenwechsel vollständig vollzogen ist.

\section{Implementierung: Lobby / Multi Scene Manager}
\label{Lobby Manager Implementierung}

Das Spielkonzept beinhaltet wie im Teil der \hyperref[Spielidee]{Spielidee} und \hyperref[Architektur]{Architektur} des Prototyps beschrieben auch eine Lobbymechanik. Die Ideen aus dem Konzept des Lobby / Multi Scene Managers wurden aufgrund der geringen Anforderungen des Prototyps ebenfalls in der Klasse GameNetworkManager.cs untergebracht. Dies hat den Vorteil, dass man die Lobby-spezifischen Variablen direkt mit den überschriebenen virutellen Methoden der Mirror Klasse "NetworkManager" kombinieren kann.

Im folgenden Beispiel werden die server-internen Variablen aufgeführt, welche im Prototyp für die Szenen-Übergreifenden Lobby-Verwaltung verantwortlich sind:

"List<GameRule> gameRules" wird verwendet, um die für diesen Server-Prozess konfigurierten Spielregeln zu verwalten. Diese sind durch den Leader einer Lobby konfigurierbar. Beispiele für konfigurierbare Spielregeln sind: 
Total Game Time - Nach wieviel Sekunden wird das Spiel automatisch zu Gunsten der Spieler entschieden, welche zum Team "Seeker" gehören.
Player Base Speed - Die Grundgeschwindigkeit der Spieler-Avatare.
Daytime - Einstellung für den Wechsel zwischen Tag - und Nacht.

"LobbyPlayer leader" beinhaltet das Spieler-Objekt, welches der Server als Leader einer Lobby ansieht.

"Dictionary<NetworkConnection, LobbyPlayer> lobbyPlayers" in diesem Dictionary wird pro Spieler die dazu gehörige "NetworkConnection" Objekt und das "LobbyPlayer" Spielerobjekt verwaltet. "NetworkConnection" ist eine Mirror-spezifische Klasse, welche alle Netzwerk-relevanten Informationen zu einem Spieler verwaltet (Bspw. die IP Adresse oder eine für diesen Spieler einzigartige Verbindungs-ID).

"bool isLobbyReady" ist ein boolscher Wert, welcher nur dann den Wert "true" annimmt, sobald alle Spieler durch das drücken auf einen "Ready"-Button signalisiert haben, dass sie bereit für den Wechsel in die Spiel-Szene sind.

"uint maxPlayers" ist ein Wert, welcher bestimmt wieviele Spieler maximal einer Lobby beitreten dürfen.

"uint keepLobbyAliveTimeInSeconds" ist ein Wert, welcher festlegt nach wie viel Sekunden Inaktivität der Spieler ein Server-Prozess beendet wird.

\begin{lstlisting}[caption= GameNetworkManager.cs Lobby Variables]
	public List<GameRule> gameRules = new List<GameRule>();
	
	private LobbyPlayer leader = null;
	
	private Dictionary<NetworkConnection, LobbyPlayer> lobbyPlayers = 
	new Dictionary<NetworkConnection, LobbyPlayer>();
	
	private bool isLobbyReady = false;
	
	private uint maxPlayers = 10;
	private uint keepLobbyAliveTimeInSeconds = 600;
\end{lstlisting}

Die folgenden Methoden sorgen dafür, dass vor dem Übergang in die Spiel-Szene die nun nicht mehr benötigten Lobby-Spielerobjekte zerstört werden und im Anschluss der gewünschte Übergang in die Spiel-Szene erfolgt. Hierfür stellt Mirror die Methode ServerChangeScene() bereit welche dafür sorgt, dass der Szenenwechsel wohl beim Server, als auch bei allen verbundenen Clients vollzogen wird.

\begin{lstlisting}[caption= GameNetworkManager.cs StartGame]
public void StartGame()
{
	clearLobbyData();
	ServerChangeScene(gameScene);
}

private void clearLobbyData()
{
	foreach(LobbyPlayer lobbyPlayer in lobbyPlayers.Values)
	{
		NetworkServer.Destroy(lobbyPlayer.gameObject);
	}
}
\end{lstlisting}

Der Aufruf der Initialisierung des Szenenwechsels zurück in die Menü-Szene geschieht aus Komplexitätsgründen aus dem InGameProgressManager heraus. Die nähere Erläuterung folgt in der Sektion \hyperref[Progress Manager]{Implementierung: Progress / Game-State Manager}

\section{Implementierung: Client Connection Manager}

Die Prinzipien des Client Connection Manager Konzepts wurden im Prototyp innerhalb der Klasse "ConnectionScript.cs" realisiert. Die Hauptaufgabe ist der Kommunikationsaustausch mit der Matchmaking API sowie der Verbingdungsaufbau zu einer Mirror-Server Instanz. Für diesen Kommunikationsaustausch wurde das Open Source Package "Rest Client" von Juan Nicholls (proyecto26) benutzt \cite{GitHub.10.02.2022}.

Im folgenden Code Beispiel wird zunächst das Datenmodell einer Lobby aus der Sicht der Matchmaking API und dem ConnectionScript aufgezeigt. Eine kurze Erläuterung der einzelenen Felder:

- "bool isPublic" sagt aus, ob eine Lobby öffentlich bereitgestellt wurde. Falls ja, können andere Spieler auch ohne Zugangs-Code über die "Join Random Lobby" Funktion des Prototyps dieser Lobby beitreten.

- "string adress" gibt die IP Adresse des Servers an, auf dem die der Mirror-Serverprozess zu dieser Lobby gestartet wurde.

- "int port" gibt den Port des Mirror-Serverprozesses an, den der Server für diesen Prozess reserviert hat.

- "string code" enthält einen Zugangsschlüssel, den Clients nutzen können, um im Austausch mit der Matchmaking API das passende Lobby Objekt zu erhalten, mit dem sie sich dann (mithilfe des adress + port Felds) ebenfalls auf den gleichen Mirror-Serverprozess verbinden können.

\begin{lstlisting}[caption= ConnectionScript.cs Matchmaking Data]
// The Data which is necessary for describing a lobby. This Information is synchronized between the REST API and the Mirror Server & Client
private class Lobby
{
	public bool isPublic;
	public string address;
	public int port;
	public string code;
}

private class PublicLobbys
{
	public Lobby[] lobbys;
}
\end{lstlisting}

Sollte ein Spieler im Hauptmenü auf den Button "Create Lobby" drücken, so wird eine POST Anfrage an die Matchmaking API geschickt, diese liefert ein neu erzeugtes Lobby-Objekt zurück, mit welchem sich der Spieler auf den neu erzeugten Mirror-Serverprozess verbindet. Folgender Code zeigt die Implementierung:

\begin{lstlisting}[caption= ConnectionScript.cs createLobby()]
private void createLobby()
{
	RestClient.Post<Lobby>(apiBaseUrl + "/v0/lobbies", "{\"isPublic\": true}").Then(lobby =>
	{
		string[] splitAddress = lobby.address.Split(new string[] { "://" }, StringSplitOptions.None);
		
		GameNetworkManager.singleton.StartClient(new UriBuilder(splitAddress[0], splitAddress[1], lobby.port).Uri);
	}).Catch(error =>
	{
		onFailure();
		Debug.LogError("Error when creating lobby: " + error);
	});
}	
\end{lstlisting}

Im Fall, dass ein Spieler auf den Button "Join Lobby" drückt, wird die Zeichenkette, welche der Nutzer in das entsprechende Code-Input Feld eingegeben hat, der Methode joinWithCode() übergeben. Diese überprüft zunächst, ob der Spieler keinen Code eingegeben hat. Falls die übergebene Zeichenkette nicht leer ist, führt das ConnectionScript eine GET Anfrage an die Matchmaking API aus, welche intern überprüft, ob eine Lobby mit diesem Zugangsschlüssel existiert. Falls dies der Fall ist, verbindet sich der Spieler automatisch mit der entsprechenden Lobby. 

\begin{lstlisting}[caption= ConnectionScript.cs joinWithCode()]
private void joinWithCode(string code)
{
	registerCallbacks();
	if (code == "")
	{
		noCodeEnteredFailure();
		return;
	}
			
	RestClient.Get<Lobby>(GameNetworkManager.singleton.frapiBaseUrl + "/v0/lobbies/" + code).Then(lobby =>
	{
		string[] splitAddress = lobby.address.Split(new string[] { "://" }, StringSplitOptions.None);
		GameNetworkManager.singleton.StartClient(new UriBuilder(splitAddress[0], splitAddress[1], lobby.port).Uri);
	}).Catch(error =>
	{
		Debug.LogError("Error when joining lobby: " + error);
		onFailure();
	}
}
\end{lstlisting}

In den beiden obrigen Beispielen wird beim Auslösen des Catch-Falls die Methode onFailure() aufgerufen. Dieser Fall tritt bei verschiedenen Szenarien ein. Beispielsweise wenn die Komminikation mit der REST API nicht möglich ist, oder diese einen 4XX oder 5XX HTTP-Status-Code liefert. Die Methode sorgt dafür, dass der Nutzer zurück ins Main-Menü geschickt wird, und ihm eine Nachricht angezeigt wird, dass die Verbindungsversuch gescheitert ist.

\begin{lstlisting}[caption= ConnectionScript.cs onFailure()]
private void onFailure()
{
	unregisterCallbacks();
	GameNetworkManager.singleton.curNetworkMessage.message = "Network Error: Could not Connect to Server";
	changeToMainMenu(true);
}

\end{lstlisting}

Die Implementierung einer onSuccess Funktion (Das Gegenstück zu onFaliure()) innerhalb des Connection-Managers ist nicht nötig, da bei erfolgreicher Verbindungsherstellung Mirror-spezifische Events und Handler-Funktionen für die weitere Logik sorgen. Diese sind nicht mehr Teil des ConnectionScript.

\section{Implementierung: Serverside Client Manager}

Das Konzept des Serverside Client Managers wird nahezu vollständig durch Mirror-interne Prozesse gelöst. Die Verwaltung aller Clients geschieht durch die "GameNetworkManager" Klasse, welche von der Mirror Klasse "NetworkManager" erbt. Die Superklasse NetworkManager wiederum nutzt eine andere Mirror-Klasse "NetworkServer", diese beinhaltet ein Dictionary, welches die Verbindungen zu allen Clients verwaltet. Im folgenden Code ist das Dictionary, sowieso die dazu gehörenden Methoden AddConnection() und RemoveConnection() zu sehen.

\begin{lstlisting}[caption= Mirror Class NetworkServer.cs Connection Handling]
/// <summary>
/// A list of local connections on the server.
/// </summary>
public static Dictionary<int, NetworkConnectionToClient> connections = new Dictionary<int, NetworkConnectionToClient>();

/// <summary>
/// <para>This accepts a network connection and adds it to the server.</para>
/// <para>This connection will use the callbacks registered with the server.</para>
/// </summary>
/// <param name="conn">Network connection to add.</param>
/// <returns>True if added.</returns>
public static bool AddConnection(NetworkConnectionToClient conn)
{
	if (!connections.ContainsKey(conn.connectionId))
	{
		// connection cannot be null here or conn.connectionId
		// would throw NRE
		connections[conn.connectionId] = conn;
		conn.SetHandlers(handlers);
		return true;
	}
	// already a connection with this id
	return false;
}


/// <summary>
/// This removes an external connection added with AddExternalConnection().
/// </summary>
/// <param name="connectionId">The id of the connection to remove.</param>
/// <returns>True if the removal succeeded</returns>
public static bool RemoveConnection(int connectionId)
{
	return connections.Remove(connectionId);
}
\end{lstlisting}

In der aktuellen Version des Prototyps werden spielerbezogene "Sonderinformationen" wie Spielername und Rolle im GameNetworkManager  mitverwaltet. Die Aufgabe des Serverside Client Managers wird in der Komponente, welche in die der Sektion \hyperref[Lobby Manager Implementierung]{Implementierung: Lobby / Multi Scene Manager} beschrieben wird, mitverwaltet. 

Diese Tatsache ist allerdings an die Randbedingungen geknüpft, dass für das den Prototypen das Mirror Framework genutzt wurde, welches viele Funktionen auf die (überschriebene) NetworkManager Klasse abwälzt, sowie dass der Prototyp keine große Anzahl an "Sonderinformationen" (wie Charakterindividualisierung, Account-Informationen / In-Game-Shop Daten etc.) speichern muss. Da der Komplexitätsaufwand für eine gesonderte Handhabung dieser Informationen bei steigender Anforderungen enorm zunimmt, ist es ratsam die Komponente "Serverside Client Manager" gesondert zu implementieren.

\section{Implementierung: Prepare-Game-Manager}

// TODO: Add Content

\section{Implementierung: Progress / Game-State Manager}
\label{Progress Manager}
// TODO: Add Content

\section{Implementierung: Runtime Spawn Manager}

// TODO: Add Content

\section{Implementierung: Interest Manager}

// TODO: Add Content


% 3. Implementierung 1 :

% Genutzte Frameworks / Technologien

% Grundlagen des Mirror Frameworks:
% --> worauf basiert es eigendlich? 
% --> Auszug Features | diese erklären.
% --> Network-Manager
% --> NetworkManager Callbacks
% --> Network Idendity / Network Behavior / Network Transform
%  --> Network Behavior Callbacks
% --> Server und Clientcode in einer Datei
% --> Dedicated Server vs. Self-Host
% --> Synchronization
% --> Remote Actions
% --> Player Game Objects
% --> Anticheat
% --> Transports






