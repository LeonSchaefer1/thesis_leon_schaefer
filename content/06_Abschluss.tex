\chapter{Abschluss}
\label{sec:abschluss}

\section{Zusammenfassung}

In dieser Arbeit wurde aufgezeigt, wo die Probleme für angehende Entwickler von Multiplayer-Spielen liegen. Es folgt eine Einordnung in die aktuelle Landschaft von Multiplayer-Spielen. Anschließend werden Grundlagen erläutert, welche benötigt werden, um die Konzepte und Realisierung genauer zu verstehen. Es wurden abstrakte Konzepte aufgestellt, welche eine Hilfestellung bei konkreten Entwicklungsproblemen geben soll. Als 'Proof of Concept' wurde ein Prototyp entwickelt, welcher diese Konzepte implementiert. Zum Abschluss wurde ein Studienkonzept erarbeitet, welches bei Durchführung beweist, dass die vorgestellten Konzepte einen tatsächlichen Mehrwert für die Entwicklung von Multiplayer-Spielen auch bei verschiedenen Arten von Projekten bringen.

\section{Fazit}

Die in dieser Arbeit gestellten Forschungsfragen \hyperref[RQ1]{RQ1} und \hyperref[RQ3]{RQ3} konnten teilweise beantwortet werden. \\
\hyperref[RQ1]{RQ1:} \\
Durch die Aufstellung der Konzepte, und anschließenden Umsetzung eines Prototyps kann nicht final beurteilt werden, ob die Probleme für eine Vielzahl an angehenden Entwicklern vereinfacht wurden. Allerdings war die Entwicklung des in dieser Arbeit beschriebenen Prototyps durch die Einhaltung der Konzepte deutlich vereinfacht. Dies könnte ein Hinweis darauf sein, dass Entwickler mit anderen Spielideen, welche ähnliche Voraussetzungen besitzen wie die des Prototyps, welcher im Rahmen dieser Arbeit entwickelt wurde, einen Vorteil aus den Konzepten ziehen könnten. Für Spielideen, welche aus einem anderen Genre kommen, eine andere Form des Matchmakings oder Art des Hostings unterstützen, kann jedoch ohne Ausführung einer weitergehenden Studie keine Aussage getroffen werden.

\hyperref[RQ3]{RQ3:} \\
Ein Studienkonzept könnte so aussehen, wie es in der Sektion \hyperref[studienkonzept]{'Studienkonzept'} beschrieben ist. Bestimmte Details können je nach konkreter weitergehender Forschungsfrage abgeändert oder ergänzt werden. Grundsätzlich führt eine korrekte Durchführung des Studienkonzepts zur Beantwortung der Forschungsfrage: \\
'Bringen die in dieser Arbeit beschriebenen Konzepte einen Mehrwert bei der Entwicklung eines Multiplayer-Spiels für angehende Entwickler?'

\section{Nächste Schritte}

Um zu beweisen, dass die Konzepte praxistauglich und für eine Vielzahl an unterschiedlichen Projekten einsetzbar sind, muss eine weitergehende Studie durchgeführt werden, welche sich an dem in dieser Arbeit beschriebenen \hyperref[studienkonzept]{Studienkonzept} orientiert. Aus den Ergebnissen der Studie können weitere Schlussfolgerungen gezogen werden, die eine abschließende Bewertung dieser Arbeit ermöglichen. 

Ebenso können andere Aspekte der Multiplayer-Spiele-Entwicklung generalisiert, und daraus weitere Konzepte abgeleitet werden, die sich am abstrakten Vorbild der hier aufgestellten Konzepte orientieren. Diese müssten das gleiche Prozedere durchlaufen. Es müsste also zunächst ein Prototyp erstellt, und im Anschluss eine Studie durchgeführt werden, welcher sich am \hyperref[studienkonzept]{Studienkonzept} dieser Arbeit orientiert.

\section{Ausblick}

Sollte das Resultat der weiterführenden Studie aufzeigen, dass die hier aufgeführten Konzepte den erhofften Mehrwert bei einer Vielzahl an Projekten bringen, so wäre es möglich die Konzepte für die Praxislehre zu nutzen. Sie könnten eine Basis bilden, um angehende Entwickler in ihrem ersten Projekt zu unterstützen. Ebenso ist es denkbar aus den Konzepten ein Software-Framework zu konzipieren und zu implementieren, welches Entwickler aktiv benutzen können, um die jeweiligen Use Cases in ihrem Projekt zu realisieren. Dieses Software-Framework wäre unabhängig von einer Engine oder einem Genre.