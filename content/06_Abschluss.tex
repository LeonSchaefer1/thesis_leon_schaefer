\chapter{Abschluss}
\label{sec:abschluss}

// TODO Add Content
- Zusammenfassung

- Diskussion

- Nächste Schritte

\section{Fazit}

Die in dieser Arbeit gestellten Forschungsfragen \hyperref[RQ1]{RQ1} und \hyperref[RQ3]{RQ3} konnten teilweise beantwortet werden. \\
\hyperref[RQ1]{RQ1:} \\
Durch die Aufstellung der Konzepte, und anschließenden Umsetzung eines Prototyps kann nicht final beurteilt werden, ob die Probleme für eine Vielzahl an angehenden Entwicklern vereinfacht wurden. Allerdings war die Entwicklung des in dieser Arbeit beschriebenen Prototyps durch die Einhaltung der Konzepte deutlich vereinfacht. Dies könnte ein Hinweis darauf sein, dass Entwickler mit anderen Spielideen, welche ähnliche Voraussetzungen besitzen wie die des Prototyps, welcher im Rahmen dieser Arbeit entwickelt wurde, einen Vorteil aus den Konzepten ziehen könnten. Für Spielideen, welche aus einem anderen Genre kommen, eine andere Form des Matchmakings oder Art des Hostings unterstützen, kann jedoch ohne Ausführung einer weitergehenden Studie keine Aussage getroffen werden.

\hyperref[RQ3]{RQ3:} \\
Ein Studienkonzept könnte so aussehen, wie es in der Sektion \hyperref[studienkonzept]{'Studienkonzept'} beschrieben ist. Bestimmte Details können je nach konkreter weitergehender Forschungsfrage abgeändert oder ergänzt werden. Grundsätzlich führt eine korrekte Durchführung des Studienkonzepts zur Beantwortung der Forschungsfrage: \\
'Bringen die in dieser Arbeit beschriebenen Konzepte einen Mehrwert bei der Entwicklung eines Multiplayer-Spiels für angehende Entwickler?'

\section{Ausblick}

Abschließend ist festzuhalten, dass diese Arbeit einen Ansatz bietet, welcher zur Erleichterung des Einstiegs in die Entwicklung von Multiplayer-Spielen führen könnte. Um zu beweisen, dass die Konzepte praxistauglich und für eine Vielzahl an unterschiedlichen Projekten einsetzbar sind, muss jedoch eine weitergehende Studie durchgeführt werden, welche sich an dem in dieser Arbeit beschriebenen \hyperref[studienkonzept]{Studienkonzept} orientiert. Aus den Ergebnissen der Studie können weitere Schlussfolgerungen gezogen werden, die eine abschließende Bewertung dieser Arbeit ermöglichen.

// TODO Nächste Schritte hinzufügen.